\documentclass[a4paper,14pt]{extarticle} 
\usepackage[a4paper,top=1.5cm, bottom=1.5cm, left=2cm, right=1cm]{geometry}
%\usepackage[T2A]{fontenc}
%\usepackage[english, russian]{babel}
\usepackage{graphicx}
\DeclareGraphicsExtensions{.pdf,.png,.jpg}
\usepackage{fontspec}
\setmainfont{Times New Roman}
\setsansfont{FreeSans}
\setmonofont{FreeMono}
\renewcommand{\baselinestretch}{1.5}
\usepackage{polyglossia}
\setdefaultlanguage{russian}
\setotherlanguages{english,russian}
\usepackage{setspace}
\usepackage[many]{tcolorbox}
\usepackage{listings}
\usepackage{multicol}
\usepackage{xcolor}
\usepackage{pdfpages}

\definecolor{codegreen}{rgb}{0,0.6,0}
\definecolor{codegray}{rgb}{0.5,0.5,0.5}
\definecolor{codepurple}{rgb}{0.58,0,0.82}
\definecolor{backcolour}{rgb}{0.95,0.95,0.92}

\lstdefinestyle{mystyle}{
    backgroundcolor=\color{backcolour},   
    keywordstyle=\color{magenta},
    numberstyle=\tiny\color{codegray},
    stringstyle=\color{codepurple},
    basicstyle=\ttfamily\footnotesize,
    breakatwhitespace=false,         
    breaklines=true,                 
    captionpos=b,                    
    keepspaces=true,                 
    numbers=left,                    
    numbersep=5pt,                  
    showspaces=false,                
    showstringspaces=false,
    showtabs=false,                  
    tabsize=2
}

\lstset{style=mystyle}

\begin{document}
    \begin{center}
        \thispagestyle{empty}
        \begin{singlespace}
        МИНИСТЕРСТВО ЦИФРОВОГО РАЗВИТИЯ, СВЯЗИ И МАССОВЫХ КОММУНИКАЦИЙ РОССИЙСКОЙ ФЕДЕРАЦИИ

        ФЕДЕРАЛЬНОЕ ГОСУДАРСТВЕННОЕ БЮДЖЕТНОЕ ОБРАЗОВАТЕЛЬНОЕ

        УЧРЕЖДЕНИЕ ВЫСШЕГО ОБРАЗОВАНИЯ

        «САНКТ-ПЕТЕРБУРГСКИЙ ГОСУДАРСТВЕННЫЙ УНИВЕРСИТЕТ ТЕЛЕКОММУНИКАЦИЙ ИМ. ПРОФ. М.А. БОНЧ-БРУЕВИЧА»

        (СПбГУТ)
        \end{singlespace}
        \vspace{-1ex}
        \rule{\textwidth}{0.4pt}
        \vspace{-5ex}

        Факультет \underline{Инфокоммуникационных сетей и систем}

        Кафедра \underline{Защищенных систем связи}
        \vspace{10ex}

        \textbf{Проектная документация вычислительной сети}
        


    \end{center}
    \vspace{4ex}
    \begin{flushright}
    \parbox{10 cm}{
    \begin{flushleft}
        Выполнил студент группы ИКТЗ-83:

        \underline{Громов А.А.} \hfill 

        \footnotesize \textit{ (Ф.И.О., № группы)} \hfill \rule[-0.85ex]{0.1\textwidth}{0.6pt}
        
        \hfill \textit{(подпись)} \normalsize

        Проверил:

        \underline{Казанцев А.А.} \hfill \rule[-0.85ex]{0.1\textwidth}{0.6pt}

        (\footnotesize \textit{уч. степень, уч. звание, Ф.И.О.) \hfill (подпись)} \normalsize

    \end{flushleft}
    }
    \end{flushright}
    \begin{center}
        \vfill
        Санкт-Петербург

        2021

    \end{center}
    \newpage

    \begin{enumerate}
        \item \textbf{Основание для разработки документации}\par
        Основанием для разработки данной проектной документации является \linebreak 
        техническое задание владельца бизнес-центра.
        \item \textbf{Перечень исходных данных}\par
        Локальная вычислительная сеть должна разместиться в 5-этажном здании \linebreak офисного назначения. Предполагается, что 
        кабельные трассы на этажах будут проходить в кабельных лотках, а межэтажные линии связи будут проложены в специальных 
        шахтах, предусмотренных проектом здания. Никаких других сетей связи в бизнес-центре не смонтировано.
        \item \textbf{Перечень нормативных документов}\par
        \begin{enumerate}
            \item ГОСТ 23678-79\par
            Каналы передачи данных. Параметры контроля и требования к цепям стыка
            \item ГОСТ 24402-88\par
            Телеобработка данных и вычислительные сети. Термины и определения
            \item ГОСТ 28907-91\par
            Системы обработки информации. Локальные вычислительные сети. \linebreak Протокол и услуги уровня управления логическим звеном данных
            \item ГОСТ 29099-91\par
            Сети вычислительные локальные. Термины и определения
            \item ГОСТ 34.913.4-91\par
            Информационная технология. Локальные вычислительные сети. Метод маркерного доступа к шине и спецификация физического уровня
            \item ГОСТ 34.936-91\par
            Информационная технология. Локальные вычислительные сети.\linebreak Определение услуг уровня управления доступом к среде
            \item ГОСТ Р ИСО/МЭК 10038-99\par
            Информационная технология. Передача данных и обмен информацией между системами. Локальные вычислительные сети. Мосты на подуровне управления доступом к среде
        \end{enumerate}
        \item \textbf{Характеристика объекта}\par
        В качестве объекта выступает бизнес-центр, расположенный в г. Санкт-Петербург. Планировка 2-5 этажей однотипна, и 
        включает в себя 16 офисных и 3 технических помещения, а также отсек с 2 лифтами и лестницей. На первом этаже расположены 
        6 помещений с возможностью оборудования отдельных выходов, а также ресепшн и 5 технических помещений. В каждое из упомянутых 
        помещений организован кабельный ввод в виде круглого отверстия в стене. Ввод магистральных линий связи выполнен в виде 
        замурованной в пол трубы, связывающих одно из технических помещений с кабельной канализацией магистрального провайдера. 
        \item \textbf{Основные проектные решения}\par
        \begin{enumerate}
            \item Требуемая ёмкость присоединения к сети связи общего пользования - 10Гбит/с.
            \item В составе ЛВС находятся коммутаторы, маршрутизатор, а также оптоволоконные и медные линии связи.
            \item Обоснованием требуемого канала является необходимость обеспечивать требуемые скорости арендаторам бизнес-центра.
            \item Для присоединения к ССОП применяется оптическая линия связи. Такая линия позволит обеспечить большую дальность 
            до точки присоединения, а также упростит дальнейшее наращивание пропускной способности каналов связи.
            \item Точка присоединения к ССОП находится в служебном помещении провайдера на 1 этаже БЦ.
            \item Для подсчета трафика используется маршрутизатор Eltex ESR-1511. Его применение обосновано гибкими возможностями 
            по контролю за трафиком и интеграции в различные системы мониторинга, например, SNMP.
            \item Поскольку монтаж маршрутизатора происходит в том же помещении, где и осуществляется присоединение к ССОП, то для 
            присоединения требуется только установка кабельной трассы между оборудованием ЛВС и провайдера.
            \item Дальше идут принципиальные схемы, на них походу не смотрим
        \end{enumerate}
        \item \textbf{Электроснабжение и электробезопасность}\par
        Электроснабжение оборудования осуществляется с помощью подведения к местам монтажа сети 220В. Требуемая мощность электросетей - 
        60Вт на каждый коммутатор и 250Вт на маршрутизатор. Резервирование питания осуществляется с помощью аккумуляторов (для 
        коммутаторов) или дополнительной линии питания для маршрутизатора.
        \item \textbf{Требования к монтажу и эксплуатации}\par
        Монтаж коммутаторов и маршрутизатора осуществляется в специальные шкафы, крепящиеся на пол (в случае маршрутизатора) или на стену 
        (в случае коммутаторов). Кабельные лотки монтируются на стены с помощью трех шурупов на высоте, достаточной для обслуживания, не 
        недостаточной для легкого доступа злоумышленников. При прокладке медных кабелей необходимо учитывать требования стандарта IEEE 802.3ab 
        по длине кабеля и его конструкции. Необходимо учитывать дополнительные 5 метров кабеля, которые могут добавить арендаторы БЦ для 
        подключения своего сетевого оборудования. Прокладка оптических кабелей осуществляется в предусмотренных проектом здания кабельных колодцах 
        и с учётом требований стандарта 10GBASE-R. Дополнительно следует учитывать наличие розеток ~220В в местах монтажа оборудования и оконечных 
        точек подключения.
        \item \textbf{Мероприятия по защите окружающей среды}\par



    \end{enumerate}

\end{document}
